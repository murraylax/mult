\documentclass[12pt]{article}
\usepackage[T1]{fontenc}
\usepackage{calc}
\usepackage{setspace}
\usepackage{multicol}
\usepackage{fancyheadings}

\usepackage{graphicx}
\usepackage{color}
\usepackage{rotating}
\usepackage{harvard}
\usepackage{aer}
\usepackage{aertt}
\usepackage{verbatim}

\setlength{\voffset}{0in}
\setlength{\topmargin}{0pt}
\setlength{\hoffset}{-0.5in}

\setlength{\oddsidemargin}{0pt}
\setlength{\headheight}{0pt}
\setlength{\headsep}{0.4in}
\setlength{\marginparsep}{0pt}
\setlength{\marginparwidth}{0pt}
\setlength{\marginparpush}{0pt}

\setlength{\footskip}{.1in}
\setlength{\textwidth}{7in}
\setlength{\textheight}{9in}
\setlength{\parskip}{0pc}

\renewcommand{\baselinestretch}{1.4}

\newcommand{\bi}{\begin{itemize}}
\newcommand{\ei}{\end{itemize}}
\newcommand{\be}{\begin{enumerate}\setlength{\leftmargin}{0pt}}
\newcommand{\ee}{\end{enumerate}}
\newcommand{\bd}{\begin{description}}
\newcommand{\ed}{\end{description}}
\newcommand{\prbf}[1]{\textbf{#1}}
\newcommand{\prit}[1]{\textit{#1}}
\newcommand{\beq}{\begin{equation}}
\newcommand{\eeq}{\end{equation}}
\newcommand{\beqa}{\begin{eqnarray}}
\newcommand{\eeqa}{\end{eqnarray}}
\newcommand{\bdm}{\begin{displaymath}}
\newcommand{\edm}{\end{displaymath}}
\newcommand{\script}[1]{\begin{cal}#1\end{cal}}
\newcommand{\citee}[1]{\citename{#1} (\citeyear{#1})}
\newcommand{\h}[1]{\hat{#1}}
\newcommand{\ds}{\displaystyle}

\newcommand{\app}
{
\appendix
}

\newcommand{\appsection}[1]
{
\let\oldthesection\thesection
\renewcommand{\thesection}{Appendix \oldthesection}
\section{#1}\let\thesection\oldthesection
\renewcommand{\theequation}{\thesection\arabic{equation}}
\setcounter{equation}{0}
}

\pagestyle{fancyplain}
\lhead{}
\chead{CBA Proposal: Fiscal and Expenditure Multipliers with Adaptive Expectations}
\rhead{\thepage}
\lfoot{}
\cfoot{}
\rfoot{}

\begin{document}
\thispagestyle{empty}
\begin{center}
\textbf{CBA FACULTY/ACADEMIC STAFF PROPOSAL FUND APPLICATION}\\
\textbf{COVER SHEET}\\
\end{center}

PROPOSAL TITLE: \begin{center}Fiscal and Expenditure Multipliers When There are Adaptive Expectations\end{center}

\be
\item Primary Applicant:
  \be
  \item Name: James Murray
  \item Department: Economics
  \item Campus Address: 413 Wimberly Hall
  \item Campus Telephone: (608) 785-5140
  \item Job title: Assistant Professor
  \item Years of service at UW-L: 2 years.
  \ee
  Additional Applicants: None.
\item Total amount requested of the committee: \$5,000.
\item Has this of will this proposal be submitted to another potential funding source? No.
\ee
  
\newpage
\noindent \textbf{Abstract:} Using linearized dynamic macroeconomics models with rational expectations is the most common modeling strategy in the macroeconomics and monetary economics literature for answering questions related to the causes of business cycle fluctuations and/or the effectiveness of monetary and fiscal policy at fixing short-run problems such as recessionary and inflationary episodes.  The predictions of this model for how macroeconomic shocks (such as technology shocks, demand shocks, cost shocks, and monetary policy shocks) affect production, inflation, and interest rates are well known.  Moreover, the predicted effect a shock of a given size has on the economy, or the predicted effect a change in policy has on the economy are the same across time.  This leads to ``one-size fits all'' policy prescriptions and an inability to explain the onset of prolonged periods of volatility in inflation and production like what the U.S. experienced in the 1970s and early 1980s.  Macroeconomic shocks can have non-trivial and time-varying effects when expectations are adaptive instead of fully rational.  The purpose of this research is to explain how adaptive expectations affect the vulnerability of the economy to adverse shocks, and how the state of expectations may hinder fiscal policy aimed at fixing such problems.  The research will focus especially on fiscal and expenditure multipliers, which are measures of how vulnerable an economy can be to a shock and how effectively government policy can fix such problems, respectively.

\noindent \textit{Keywords:} Learning, expectations, fiscal policy, expenditure multipliers. \\
\noindent \textit{JEL classification:} C13, E31, E50. \\

\noindent \textbf{Objectives:}
\be
\item The purpose of this research is to expand macreconomists' knowledge of how the same size shocks or the same size government expenditure policy can have different impacts on the economy at different times, depending crucially on the state of expectations which are adaptive.  
\item Enhance my teaching of undergraduate macroeconomics by improving my understanding for the role adaptive expectations play on 1) the causes for past U.S. recessions, 2) the varying degrees of severity, and 3) the varying degrees of effectiveness of fiscal policy responses.
\ee

\noindent \textbf{Outcomes:}
\be
\item Quality research paper which will be distributed first through conference presentations and working paper series in Fall 2011 and eventually published in a peer-reviewed journal. 
\item My undergraduate macroeconomics students will be exposed to a unique set of knowledge of the role adaptive expectations play in interpreting U.S. business cycle fluctuations and fiscal policy responses.  This can be documented with a teaching narrative and a short-answer classroom assessment activity.  I expect to have results to share results with undergraduate macroeconomics students by Spring 2012.
\ee

\noindent \textbf{Impact:}
\be
\item Fiscal policy literature:  The recent U.S. recession and financial crisis has triggered a growing interest among macroeconomics researchers on the effectiveness of fiscal policy in combating recessions.  Many papers have explicitly measured the \textit{fiscal expenditure multiplier}, the most simple of which is a measure of how much total spending in the economy increases as a result of a \$1 increase in government spending.  Multipliers greater than 1.0 indicate the government can combat a recession by increasing government spending which causes an even larger increase in total spending.  This brings \textit{Gross Domestic Product (GDP)}, the total value of production of goods and services produced in the economy, closer to \textit{potential GDP}, the value of GDP when all resources are being used efficiently and unemployment is at its lowest sustainable level.  The larger is the fiscal expenditure multiplier, the more effectively and quickly the government can bring the economy back to its potential.

Coenen et. al. (2010) claim a robust finding that fiscal expenditure multipliers are greater than one over a wide range of macroeconomics models.  Other papers have disagreed, finding instances of estimated structural models that predict small or near-zero fiscal expenditure multipliers.  Such papers include Cogan et. al. (2010), Cwik and Weiland (2010), and Uhlig (2010).  Leeper, Traum, and Walker (2010) demonstrate the estimated multiplier can depend crucially on the choice of economic models.  All these papers assume rational expectations which implies the size of the multiplier remains constant over time.  The research I propose will complement this literature by demonstrating how adaptive expectations leads to time-varying multipliers, which may reveal time periods in which fiscal policy is more effective than at other times.  The present research also complements my ongoing work on the impact adaptive expectations has on business cycle fluctuations, see Murray (2010) and (2011).

Target journals include the following: \textit{Macroeconomic Dynamics}, \textit{Journal of Economic Dynamics and Control}, \textit{Studies in Nonlinear Dynamics and Econometrics}, \textit{B.E. Journal of Macroeconomics}, \textit{Quantitative Economics}, \textit{Journal of Macroeconomics}, \textit{Fiscal Studies}.

\item This research will enhance the understanding of the causes of past U.S. recessions and shed light on why the economy was able to recover quickly at times and not at other times.  Though this paper focuses on the effectiveness of fiscal policy, it will explicitly model monetary policy, since the Federal Reserve's decisions can influence adaptive expectations.  For these reasons, I will gain an \textbf{unique intuitive understanding} of the causes of business cycle fluctuations experienced in the U.S. and effective fiscal policy responses \textbf{that I can share with my students} in the following classes: ECO 120: Principles of Macroeconomics, ECO 301: Money and Banking, ECO 400: Monetary Theory and Policy, ECO 305: Intermediate Macroeconomics, BUS 712: Business Fluctuations.
\ee

\noindent \textbf{Description:}

The purpose of this research is to expand macroeconomists understanding of fiscal expenditure multipliers, specifically how multipliers can vary over time because the manner in which expectations are formed vary over time when allowing expectations to be adaptive instead of fully rational.  The policy implications are immediate: a relatively large expenditure multiplier means the government can effectively combat recessionary episodes; smaller multipliers imply the government will be less effective.  

To accomplish this objective, I will produce a high quality paper to circulate at conferences and working paper series, and eventually publish in a peer reviewed journal.  The activities necessary to accomplish these goals, and the expected time frames, are the following:
\be
\item Develop a dynamic stochastic general equilibrium model (a common framework for models in the macroeconomics literature) with adaptive expectations that describes the interdependence of relevant macroeconomic variables such as real GDP, investment, inflation, interest rates, and government spending.  The model will have the same basic framework as Smets and Wouters (2007) and Slobydan and Wouters (2007) and (2008), and is not unlike the models used in my existing working papers. \\
\textit{Time frame:} Complete by early summer 2011.
\item Write computer programs in C (C is a computer programming language) that solve and simulate the model, and can be used to estimate models.  I have previous experience successfully creating similar programs for similar models used in my existing working papers (see bibliography).\\
\textit{Time frame:} Complete by end of summer 2011.
\item Write introduction, literature review, and motivation for the paper.\\
\textit{Time frame:} Complete by early Fall 2011.
\item Use computer models to generate results, include results, descriptions, and conclusions in the paper.  At this point, all analysis needed for the paper will be complete and I will have the first complete draft of the working paper.\\
\textit{Time frame:} Complete by end of Fall 2011. 
\item Circulate paper in working paper series sponsored by SSRN (Social Science Research Network) and the Midwest Economics Association (MEA) 2012 annual conference.  I plan on organizing a session about this topic at the MEA 2012 conference.  At the MEA 2011 conference I was able to gather interest among other experts in this field to participate in such a session.\\
\textit{Time frame:} Complete by March 2012.
\item Consider feedback from the paper and submit it for publication by the beginning of Summer 2012.
\ee

\noindent \textbf{Budget:}  Stipend = \$5,000.  Total = \$5,000.

\vspace*{-0.4in}
\begin{singlespace}
\nocite{*}
\bibliographystyle{econometrica}
\bibliography{mult.bib}
\end{singlespace}

\end{document}



